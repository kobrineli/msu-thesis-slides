\documentclass[10pt]{beamer}

\usepackage[utf8]{inputenc}
% Remove russian in English presentation
\usepackage[english,russian]{babel}

\usetheme[
  sectionpage=simple,
  numbering=fraction
]{metropolis}
\usepackage{appendixnumberbeamer}

\usepackage{booktabs}
\usepackage[scale=2]{ccicons}

\usepackage{pgfplots}
\usepgfplotslibrary{dateplot}

\usepackage{xspace}
\newcommand{\themename}{\textbf{\textsc{metropolis}}\xspace}

\usepackage{todonotes}
\usepackage{ifthen}%
\providecommand\enabletodos{true}%
\ifthenelse{ \equal{\enabletodos}{true} }{%
  \presetkeys{todonotes}{inline}{}%
}{%
  \presetkeys{todonotes}{disable}{}%
}%

\setbeamertemplate{caption}{\raggedright\insertcaption\par}

\AtBeginSection[]{}

\setbeamertemplate{section in toc}{%
  \alert{$\bullet$}~\inserttocsection}
\setbeamercolor{subsection in toc}{bg=white,fg=structure}
\setbeamertemplate{subsection in toc}{%
  \hspace{1.2em}{\alert{\rule[0.3ex]{3pt}{3pt}}~\inserttocsubsection\par}}

\usepackage{hyperref}
\hypersetup{unicode=true}

\usepackage{adjustbox}
\protected\def\psverb#1{\def\innerpsverb##1#1{\texttt{##1}}\innerpsverb}
\usepackage{listings}

\usepackage{makecell}
\usepackage{changepage}

\usepackage{booktabs}
\usepackage{multirow}
\usepackage{colortbl}

\title{Символьные предикаты безопасности для обнаружения ошибок в бинарном коде}
%\subtitle{Подназвание доклада}
\date{27 апреля 2023 г.}
\author{Кобрин Илай Александрович}

% Russian
\institute{МГУ им. М.В. Ломоносова \\
Факультет вычислительной математики и кибернетики \\
Кафедра системного программирования}
%\titlegraphic{\hfill\includegraphics[height=0.5cm]{logo_isp_ru.png}}
% English
%\institute{ISP RAS}
%\titlegraphic{\hfill\includegraphics[height=0.5cm]{logo_isp_en.png}}

\definecolor{Blue}{HTML}{29619b}
\setbeamercolor{frametitle}{bg=Blue}
\setbeamercolor{palette primary}{bg=Blue}

\begin{document}

\maketitle

\begin{frame}{Актуальность}
Программное обеспечение стремительно развивается, принося с собой множество
программных ошибок, поэтому необходимо его тестировать.

Программист самостоятельно может тестировать ПО только на ограниченном наборе
тестов, поэтому необходимо разрабатывать методы автоматического тестирования.

\textbf{Динамическая символьная интерпретация} позволяет строить математическую
модель программы и находить ошибки путем решения соответствующих систем уравнений и неравенств.
%\begin{itemize}
%\end{itemize}
\end{frame}

\begin{frame}{Динамическая символьная интерпретация}
\begin{itemize}
    \item \textbf{Динамическая символьная интерпретация} --- метод автоматического тестирования
        программ, при котором происходит интерпретация программы, где конкретным
        значениям переменных, зависящих от входных данных, сопоставляются
        символьные переменные, принимающие произвольные значения.
    \item \textbf{Предикат пути} --- система из уравнений и неравенств над
        символьными переменными и константными значениями, решение которой
        обеспечивает прохождение потока управления по исследуемому пути.
    \item \textbf{Предикат безопасности} --- дополнительные условия на предикат пути,
        которые позволяют обнаружить ошибку в программе.
\end{itemize}
\end{frame}

%\begin{frame}{Построение предиката безопасности}
%\textbf{Предикат безопасности}~--- дополнительные условия на предикат пути,
%которые позволяют обнаружить ошибку в программе.
%
%\begin{center}
%\tiny
%\begin{tabular}{ l  l  l  l }\toprule
%\textbf{Символьное состояние} & \textbf{Инструкция} & \textbf{Множество формул} & \textbf{Предикат пути} \\
%\hline
%$rax = \phi_1, rbx = \phi_2$ & --- & $S = \varnothing$ & $\prod = true$ \\
%\hline
%$rax = \phi_1, rbx = \phi_3$ & \texttt{add rbx, 2} & $S = \{\phi_3 = \phi_2 +
%2\}$ & $\prod = true$ \\
%\hline
%\multirow{2}{*}{$rax = \phi_1, rbx = \phi_3$} & \texttt{cmp rbx, 2048} & \multirow{2}{*}{$S =
%\{\phi_3 = \phi_2 + 2\}$} & \multirow{2}{*}{$\prod = (\phi_3 < 2048)$} \\
%& \texttt{jge .out} &&\\
%\hline
%\multirow{2}{*}{$rax = \phi_1, rbx = \phi_3$} & \texttt{cmp rbx, 0} & \multirow{2}{*}{$S =
%\{\phi_3 = \phi_2 + 2\}$} & \multirow{2}{*}{$\prod = (\phi_3 < 2048) \land
%(\phi_3 \geq 0)$} \\
%& \texttt{jl .out} &&\\
%\hline
%$rax = \phi_1, rbx = \phi_3$ & \texttt{mov [rax + rbx], 1} &
%$S = \{\phi_3 = \phi_2 + 2, \phi_4 = \phi_1 + \phi_3\}$ & $\prod = (\phi_3 < 2048) \land (\phi_3 \geq 0)$ \\
%\midrule
%\multicolumn{4}{c}{\textbf{Предикат безопасности}} \\
%\multicolumn{4}{c}{$\prod_{security} = (\phi_3 < 2048) \land (\phi_3 \geq 0)
%\land (\phi_4 = 0)$} \\
%\bottomrule
%\end{tabular}
%\end{center}
%\end{frame}

\begin{frame}{Постановка задачи}
Необходимо разработать и реализовать метод построения и проверки символьных
предикатов безопасности для поиска ошибок в бинарном коде с помощью динамической
символьной интерпретации, который должен
\begin{itemize}
    \item Находить ошибки целочисленного переполнения, выхода за границы
        массива, деления на ноль и разыменования нулевого указателя;
    \item Для ошибок целочисленного переполнения находить источники ошибок и их
        стоки;
    \item Генерировать входные данные, которые приведут к проявлению ошибки;
    \item Для ошибок целочисленного переполнения и выхода за границу массива
        пытаться подбирать такие входные данные, чтобы последствия проявления
        ошибки были более серьезными.
\end{itemize}
\end{frame}

\begin{frame}{Известные решения}
Разработанный метод был реализован на базе инструмента Sydr, использующего
Triton и DynamoRIO.

Примеры других известных решений:
\begin{itemize}
    \item KLEE --- поиск ошибок одновременно с расширением покрытия
    \item Mayhem --- автоматическая генерация эксплойтов
    \item Google Sanitizers --- обнаружение ошибок в программе
        во время ее работы
    \item SAVIOR --- фреймворк для гибридного тестирования, нацеленный на поиск
        ошибок
    \item ParmeSan --- инструмент для фаззинга с обратной связью по санитайзерам
    \item IntScope --- инструмент для поиска ошибок целочисленного переполнения
        при помощи символьной интерпретациии
\end{itemize}
\end{frame}

\begin{frame}{Схема проверки предиката безопасности}
\begin{itemize}
    \item Интерпретируем программу, анализируем каждую инструкцию на пути
        выполнения
    \item Составляем предикат безопасности для обнаружения конкретной ошибки
    \item Конъюнкция предиката безопасности и предиката пути
    \item Проверка выполнимости конъюнкции с помощью математического решателя
    \item Если выполнима --- печатаем предупреждение и сохраняем файл
        для воспроизведения ошибки
    \item Для ошибок разыменования нулевого указателя и деления на ноль
        составляем уравнения на равенство адреса и делителя нулю
\end{itemize}
\end{frame}

%\begin{frame}{Разыменование нулевого указателя и деление на ноль}
%\begin{itemize}
%    \item Простые, но распространенные в ПО ошибки.
%    \item Для ошибок разыменования нулевого указателя ищем инструкции,
%        разыменовывающие адрес. Составляем уравнение на равенство символьного
%        адреса нулю.
%    \item Для инструкций деления составляем уравнение на равенство символьного
%        делителя нулю.
%\end{itemize}
%\end{frame}

\begin{frame}{Выход за границу массива}
\begin{itemize}
    \item Строим предикат безопасности для инструкций, разыменовывающих адрес.
    \item Находим границы с помощью теневой кучи, теневого стека или с помощью
        эвристического подхода.
    \item Составляем предикат безопасности в виде неравенства на то, может ли
        адрес выходить за границы.
    \item Составляем дополнительные условия, чтобы попытаться перезаписать адрес
        возврата из функции или разыменовать отрицательный адрес.
\end{itemize}
\end{frame}

\begin{frame}{Целочисленного переполнение}
\begin{itemize}
    \item Строим предикат безопасности для арифметических инструкций,
        представляющий из себя равенство флагов \texttt{CF}/\texttt{OF} единице.
    \item Так как арифметики в программах очень много, проверяем предикат
        безопасности только если был найден соответствующий сток ошибки.
    \item Стоками ошибки являются: условный переход, разыменование
        адреса, аргументы опасных функций (\texttt{malloc}, \texttt{memcpy} и
        т.п.),
        аргументы остальных функций.
    \item Знаковость переполнения определяем по ранее встретившимся инструкциям
        условных переходов.
    \item Для функций копирования и функций выделения памяти составляем предикат
        безопасности так, чтобы ошибка переполнения вероятнее привела к
        дальнейшему выходу за границы массива.
\end{itemize}
\end{frame}

\begin{frame}{Оценка точности на наборе тестов Juliet}
\footnotesize
\begin{adjustwidth}{-2.5em}{-2.5em}
\begin{tabular}{ l r | r r >{\columncolor[gray]{0.9}}r|
r r >{\columncolor[gray]{0.9}}r}\toprule
\multirow{2}{*}{\textbf{CWE}} & \multirow{2}{*}{\textbf{P=N}} &
\multicolumn{3}{c|}{\textbf{Текстовые ошибки}} &
\multicolumn{3}{c}{\textbf{Верификация}} \\
&& \textbf{TPR} & \textbf{TNR} & \textbf{ACC} & \textbf{TPR} & \textbf{TNR} & \textbf{ACC} \\
Stack BOF & 188 & 100\% & 100\% & 100\% & 100\% & 100\% & 100\% \\
Heap BOF & 376 & 100\% & 100\% & 100\% & 100\% & 100\% & 100\% \\
Buffer Underwrite & 188 & 100\% & 100\% & 100\% & 100\% & 100\% & 100\% \\
Buffer Overread & 188 & 100\% & 100\% & 100\% & 100\% & 100\% & 100\% \\
Buffer Underread & 188 & 100\% & 100\% & 100\% & 100\% & 100\% & 100\% \\
Integer Overflow & 2580 & 99.92\% & 90.89\% & 95.41\% & 99.92\% & 90.89\% & 95.41\% \\
Integer Underflow & 1922 & 99.90\% & 91\% & 95.45\% & 99.90\% & 91\% & 95.45\% \\
Unexpected Sign Ext & 752 & 100\% & 100\% & 100\% & 100\% & 100\% & 100\% \\
Signed to Unsigned & 752 & 100\% & 100\% & 100\% & 100\% & 100\% & 100\% \\
Divide by Zero & 564 & 66.67\% & 100\% & 83.33\% & 66.67\% & 100\% & 83.33\% \\
Int Overflow to BOF & 188 & 100\% & 100\% & 100\% & 100\% & 100\% & 100\% \\
\midrule
\textbf{ИТОГО} & 7886 & 97.57\% & 94.83\% & 96.20\% & 97.57\% & 94.83\% & 96.20\% \\
\bottomrule
\end{tabular}
\end{adjustwidth}
\end{frame}

\begin{frame}{Результаты практического применения}
Были найдены ошибки в следующих проектах с открытым исходным кодом:
\begin{itemize}
    \item FreeImage (целочисленное переполнение)
    \item rizin (целочисленное переполнение, ведущее к выходу за границы
        массива)
    \item xlnt (целочисленное переполнение и выход за границу массива)
    \item unbound (целочисленное переполнение)
    \item hdp (деление на ноль)
    \item miniz (целочисленное переполнение)
\end{itemize}
\end{frame}

\begin{frame}{Заключение}
Был разработан и реализован метод построения и проверки символьных
предикатов безопасности для поиска ошибок в бинарном коде.
\begin{itemize}
    \item Метод позволяет находить ошибки разыменования нулевого указателя,
        выхода за границы массива, целочисленного переполнения и деления на ноль
    \item Метод протестирован на наборе тестов Juliet и показал высокую
        точность в 96.20\%
    \item Найдено множество ошибок в проектах с открытым исходным кодом:
        FreeImage, rizin, xlnt, unbound, hdp и miniz
\end{itemize}
\end{frame}

\begin{frame}{Публикации}
\footnotesize
Результаты были опубликованы:
\begin{itemize}
    \item Vishnyakov A., Logunova V., Kobrin E., Kuts D., Parygina D., Fedotov A. Symbolic Security Predicates: Hunt Program Weaknesses. 2021 Ivannikov ISPRAS Open Conference (ISPRAS), IEEE, 2021, pp. 76-85. DOI: 10.1109/ISPRAS53967.2021.00016
    \item Вишняков А.В., Кобрин И.А., Федотов А.Н. Поиск ошибок в бинарном коде методами динамической символьной интерпретации. Труды Института системного программирования РАН, том 34, вып. 2, 2022, стр. 25-42. DOI: 10.15514/ISPRAS-2022-34(2)-3
\end{itemize}

Публикации по теме работы:
\begin{itemize}
    \item Vishnyakov A., Fedotov A., Kuts D., Novikov A., Parygina D., Kobrin E., Logunova V., Belecky P., Kurmangaleev Sh. Sydr: Cutting Edge Dynamic Symbolic Execution. 2020 Ivannikov ISPRAS Open Conference (ISPRAS), IEEE, 2020, pp. 46-54. DOI: 10.1109/ISPRAS51486.2020.00014
    \item Vishnyakov A., Kuts D., Logunova V., Parygina D., Kobrin E., Savidov G., Fedotov A. Sydr-Fuzz: Continuous Hybrid Fuzzing and Dynamic Analysis for Security Development Lifecycle. 2022 Ivannikov ISPRAS Open Conference (ISPRAS), IEEE, 2022, pp. 111-123. DOI: 10.1109/ISPRAS57371.2022.10076861
\end{itemize}
\end{frame}

\appendix
\begin{frame}[standout] \vfill Спасибо за внимание! \vfill \end{frame}

\begin{frame}{Juliet Dynamic}
Для проверки эффективности работы предикатов была сделана тестовая система на
основе набора тестов Juliet.
\begin{itemize}
    \item Собираем и запускаем тесты, собираем результаты TP, TN, FP, FN на
        основе вывода Sydr
    \item Проверяем на основе сгенерированного файла корректность результата с
        помощью санитайзеров
    \item Выводим отдельно результаты на основе вывода Sydr и результаты,
        верифицированные санитайзерами
\end{itemize}
\end{frame}

\begin{frame}[fragile]{FreeImage}
В проекте FreeImage были найдены ошибки целочисленного переполнения в следующих
местах:

\begin{lstlisting}[language=C, basicstyle=\small\ttfamily,
                   xleftmargin=2em,
                   captionpos=b,
                   label=lst:freeimage_bmp_overflow]
unsigned off_head, off_setup, off_image, i;
...
fseek(ifp, off_setup + 792, SEEK_SET);
...
\end{lstlisting}

\begin{lstlisting}[language=C, basicstyle=\small\ttfamily,
                   xleftmargin=2em,
                   captionpos=b,
                   label=lst:freeimage_tiff_overflow]
int doff;
...
fseek(ifp, doff + base, SEEK_SET);
...
\end{lstlisting}
\end{frame}

\begin{frame}[fragile]{FreeImage}
Неявное преобразование типа \texttt{long} к \texttt{int}:
\begin{lstlisting}[language=C, basicstyle=\small\ttfamily,
                   xleftmargin=2em,
                   captionpos=b,
                   label=lst:freeimage-tiff-implicit]
int parse_tiff(int base);
...
parse_tiff(thumb_offset + 12);
\end{lstlisting}

Целочисленное переполнение в условном переходе:
\begin{lstlisting}[language=C, basicstyle=\small\ttfamily,
                   xleftmargin=2em,
                   captionpos=b,
                   label=lst:freeimage-condition-overflow]
if (*len * tagtype_dataunit_bytes[(*type <= LIBRAW_EXIFTAG_TYPE_IFD8) ? *type : 0] > 4)
{
    fseek(ifp, get4() + base, SEEK_SET);
}
\end{lstlisting}

Беззнаковое целочисленное переполнение при вычислении ширины:
\begin{lstlisting}[language=C, basicstyle=\small\ttfamily,
                   xleftmargin=2em,
                   captionpos=b,
                   label=lst:freeimage-width-overflow]
width = raw_width - left_margin - (get4() & 7);
\end{lstlisting}
\end{frame}

\begin{frame}[fragile]{rizin}
В проекте rizin была найдена ошибка целочисленного переполнения, приводящая к
выходу за границы массива:
\begin{lstlisting}[language=C, basicstyle=\small\ttfamily,
                   xleftmargin=2em,
                   captionpos=b,
                   label=lst:rizin-overflow]
symbols_size = (symbols_count + 1) * 2 * sizeof(struct symbol_t);

if (symbols_size < 1) {
    ht_pp_free(hash);
    return NULL;
}
if (!(symbols = calloc(1, symbols_size))) {
    ht_pp_free(hash);
    return NULL;
}
\end{lstlisting}
\end{frame}

\begin{frame}[fragile]{rizin}
Выход за границы массива вследствие переполнения:
\begin{lstlisting}[language=C, basicstyle=\small\ttfamily,
                   xleftmargin=2em,
                   captionpos=b,
                   label=lst:rizin-oob]
for (i = 0; i < bin->nsymtab && i < symbols_count; i++) {
...
    symbols[j].last = 0;
    if (inSymtab(hash, symbols[j].name, symbols[j].addr)) {
        RZ_FREE(symbols[j].name);
    } else {
        j++;
    }
...
}
...
symbols[j].last = true;
\end{lstlisting}
\end{frame}

\begin{frame}[fragile]{xlnt}
Ошибки целочисленного переполнения при умножении и сложении:
\tiny
\begin{lstlisting}[language=C, basicstyle=\small\ttfamily,
                   xleftmargin=2em,
                   captionpos=b,
                   label=lst:xlnt-overflow]
in_->seekg(
      static_cast<std::ptrdiff_t>(sector_data_start() +
      sector_size() * static_cast<std::size_t>(id)));
std::vector<byte> sector(sector_size(), 0);
\end{lstlisting}
\end{frame}

\begin{frame}[fragile]{xlnt}
Найденная ошибка выхода за границы массива:
\begin{lstlisting}[language=C, basicstyle=\small\ttfamily,
                   xleftmargin=2em,
                   captionpos=b,
                   label=lst:xlnt-oob]
sector_chain
compound_document::follow_chain(sector_id start,
                                const sector_chain &table)
{
    auto chain = sector_chain();
    auto current = start;
    while (current >= 0)
    {
        chain.push_back(current);
        current =
            table[static_cast<std::size_t>(current)];
    }
    return chain;
}
\end{lstlisting}
\end{frame}

\begin{frame}[fragile]{unbound}
Ошибка целочисленного переполнения в unbound:
\begin{lstlisting}[language=C, basicstyle=\small\ttfamily,
                   xleftmargin=2em,
                   captionpos=b,
                   label=lst:unbound_overflow]
int sign = 0;
uint32_t i = 0;
uint32_t seconds = 0;

for(*endptr = nptr; **endptr; (*endptr)++) {
    switch (**endptr) {
    ...
    case '9':
        i *= 10;
    ...
}
\end{lstlisting}
\end{frame}

\begin{frame}[fragile]{hdp}
Ошибка целочисленного деления на ноль в проекте hdp:
\begin{lstlisting}[language=C, basicstyle=\small\ttfamily,
                   xleftmargin=2em,
                   captionpos=b,
                   label=lst:hdp-zerodiv]
int32 buf_size;
/* we are bounded above by VDATA_BUFFER_MAX */
buf_size = MIN(total_bytes, VDATA_BUFFER_MAX);
// make sure there is at least room
// for one record in our buffer
chunk = buf_size / hsize + 1;
\end{lstlisting}
\end{frame}

\begin{frame}[fragile]{miniz}
Ошибки целочисленного переполнения в miniz (зависимость PyTorch):
\begin{lstlisting}[language=C, basicstyle=\small\ttfamily,
                   xleftmargin=2em,
                   captionpos=b,
                   label=lst:miniz-overflow]
if (cdir_size < pZip->m_total_files *
                MZ_ZIP_CENTRAL_DIR_HEADER_SIZE)
    return mz_zip_set_error(pZip,
               MZ_ZIP_INVALID_HEADER_OR_CORRUPTED);
\end{lstlisting}
\end{frame}

\begin{frame}{Ссылки на результаты}
\begin{itemize}
    \item Juliet Dynamic: https://github.com/ispras/juliet-dynamic
    \item FreeImage: https://sourceforge.net/p/freeimage/bugs/347/
    \item rizin: https://github.com/rizinorg/rizin/issues/2935
    \item xlnt: https://github.com/tfussell/xlnt/issues/616,
        https://github.com/tfussell/xlnt/issues/626
    \item unbound: https://github.com/NLnetLabs/unbound/issues/637
    \item hdp: https://bugs.launchpad.net/ubuntu/+source/libhdf4/+bug/1915417
    \item miniz: https://github.com/richgel999/miniz/pull/238
\end{itemize}
\end{frame}
\end{document}
