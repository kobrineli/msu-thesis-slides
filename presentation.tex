\documentclass[10pt]{beamer}

\usepackage[utf8]{inputenc}
% Remove russian in English presentation
\usepackage[english,russian]{babel}

\usetheme[
  sectionpage=simple,
  numbering=fraction
]{metropolis}
\usepackage{appendixnumberbeamer}

\usepackage{booktabs}
\usepackage[scale=2]{ccicons}

\usepackage{pgfplots}
\usepgfplotslibrary{dateplot}

\usepackage{xspace}
\newcommand{\themename}{\textbf{\textsc{metropolis}}\xspace}

\usepackage{todonotes}
\usepackage{ifthen}%
\providecommand\enabletodos{true}%
\ifthenelse{ \equal{\enabletodos}{true} }{%
  \presetkeys{todonotes}{inline}{}%
}{%
  \presetkeys{todonotes}{disable}{}%
}%

\setbeamertemplate{caption}{\raggedright\insertcaption\par}

\AtBeginSection[]{}

\setbeamertemplate{section in toc}{%
  \alert{$\bullet$}~\inserttocsection}
\setbeamercolor{subsection in toc}{bg=white,fg=structure}
\setbeamertemplate{subsection in toc}{%
  \hspace{1.2em}{\alert{\rule[0.3ex]{3pt}{3pt}}~\inserttocsubsection\par}}

\usepackage{hyperref}
\hypersetup{unicode=true}

\usepackage{adjustbox}
\protected\def\psverb#1{\def\innerpsverb##1#1{\texttt{##1}}\innerpsverb}
\usepackage{listings}

\usepackage{makecell}
\usepackage{changepage}

\usepackage{booktabs}
\usepackage{multirow}
\usepackage{colortbl}

\title{Поиск ошибок целочисленного переполнения методами динамической символьной
интерпретации}
%\subtitle{Подназвание доклада}
%\date{17 ноября 2021 г.}
\author{Кобрин Илай Александрович}

% Russian
\institute{МГУ им. М.В. Ломоносова, кафедра системного программирования}
%\titlegraphic{\hfill\includegraphics[height=0.5cm]{logo_isp_ru.png}}
% English
%\institute{ISP RAS}
%\titlegraphic{\hfill\includegraphics[height=0.5cm]{logo_isp_en.png}}

\definecolor{Blue}{HTML}{29619b}
\setbeamercolor{frametitle}{bg=Blue}
\setbeamercolor{palette primary}{bg=Blue}

\begin{document}

\maketitle

\begin{frame}{Введение}
\begin{itemize}
    \item \textbf{Символьная интерпретация} --- метод автоматического тестирования
        программ, при котором происходит интерпретация программы, где конкретным
        значениям переменных, зависящих от входных данных, сопоставляются
        символьные переменные, принимающие произвольные значения.
    \item \textbf{Фаззинг} --- метод автоматического тестирования программ, при котором
        программе на вход подается множество наборов входных данных, после чего
        анализируется реакция программы и генерируются новые входные данные.
    \item Совместная работа фаззинга и символьной интерпретации называется
        \textbf{гибридным фаззингом}.
\end{itemize}
\end{frame}

\begin{frame}{Построение предиката безопасности}
\textbf{Предикат безопасности}~--- дополнительные условия на предикат пути,
которые позволяют обнаружить ошибку в программе.

\begin{center}
\tiny
\begin{tabular}{ l  l  l  l }\toprule
\textbf{Символьное состояние} & \textbf{Инструкция} & \textbf{Множество формул} & \textbf{Предикат пути} \\
\hline
$rax = \phi_1, rbx = \phi_2$ & --- & $S = \varnothing$ & $\prod = true$ \\
\hline
$rax = \phi_1, rbx = \phi_3$ & \texttt{add rbx, 2} & $S = \{\phi_3 = \phi_2 +
2\}$ & $\prod = true$ \\
\hline
\multirow{2}{*}{$rax = \phi_1, rbx = \phi_3$} & \texttt{cmp rbx, 2048} & \multirow{2}{*}{$S =
\{\phi_3 = \phi_2 + 2\}$} & \multirow{2}{*}{$\prod = (\phi_3 < 2048)$} \\
& \texttt{jge .out} &&\\
\hline
\multirow{2}{*}{$rax = \phi_1, rbx = \phi_3$} & \texttt{cmp rbx, 0} & \multirow{2}{*}{$S =
\{\phi_3 = \phi_2 + 2\}$} & \multirow{2}{*}{$\prod = (\phi_3 < 2048) \land
(\phi_3 \geq 0)$} \\
& \texttt{jl .out} &&\\
\hline
$rax = \phi_1, rbx = \phi_3$ & \texttt{mov [rax + rbx], 1} &
$S = \{\phi_3 = \phi_2 + 2, \phi_4 = \phi_1 + \phi_3\}$ & $\prod = (\phi_3 < 2048) \land (\phi_3 \geq 0)$ \\
\midrule
\multicolumn{4}{c}{\textbf{Предикат безопасности}} \\
\multicolumn{4}{c}{$\prod_{security} = (\phi_3 < 2048) \land (\phi_3 \geq 0)
\land (\phi_4 = 0)$} \\
\bottomrule
\end{tabular}
\end{center}
\end{frame}

\begin{frame}{Постановка задачи}
Необходимо реализовать метод поиска ошибок целочисленного переполнения в
бинарном коде с помощью динамической символьной интерпретации, который
\begin{itemize}
    \item Строит предикат безопасности для ошибок целочисленного переполнения;
    \item Находит источники и стоки ошибок;
    \item Находит ошибки для различных размеров целочисленных типов
    \item Определяет знаковость результата арифметической операции, приводящей к
        переполнению, когда это возможно;
    \item Генерирует входные данные, приводящие к проявлению ошибки;
    \item В определенных случаях подбирает такие входные данные, чтобы
        последствия ошибки переполнения вероятнее приводили к аварийному
        завершению.
\end{itemize}
\end{frame}

\begin{frame}{Известные решения}
Разработанный метод был реализован на базе инструмента Sydr, использующего
Triton и DynamoRIO.

Примеры других известных решений:
\begin{itemize}
    \item KLEE
    \item Mayhem
    \item Google Sanitizers
    \item SAVIOR
    \item IntScope
\end{itemize}
\end{frame}

\begin{frame}{Схема проверки предиката безопасности}
\begin{itemize}
    \item Составляем предикат безопасности
    \item Применяем слайсинг к предикату пути
    \item Конъюнкция предиката безопасности и получившегося предиката пути
    \item Проверка выполнимости конъюнкции с помощью Z3
    \item Если SAT - печатаем предупреждение и сохраняем файл
\end{itemize}
\end{frame}

\begin{frame}{Понятие источника и стока}
\begin{itemize}
    \item Из-за слишком частых арифметических инструкций проверять на
        переполнение каждую слишком расточительно, а также может привести к
        большому количеству ложных срабатываний.
    \item \textbf{Источник}~--- арифметическая инструкция, в которой потенциально
        может произойти переполнение значения
    \item \textbf{Сток}~--- место, в котором используется переполненное ранее
        значение.
    \item Сначала находим потенциальный \textbf{источник}; если нашли
        \textbf{сток}, в котором он используется, проверяем на переполнение.
    \item Создаем два предиката для беззнакового и знакового переполнений, где
        проверяем равенство флагов \texttt{CF} и \texttt{OF} единице.
\end{itemize}
\end{frame}

\begin{frame}{Виды sink'ов}
    Мы выделяем следующие места, где переполнение может привести к наиболее
    неприятным последствиям:
    \begin{itemize}
        \item Условные переходы
        \item Разыменование указателя
        \item Аргументы функций
    \end{itemize}
    Особенно опасно переполнение в аргументах \texttt{malloc}, \texttt{memcpy} и
    т.п.
    Для таких функций проверяем все символьные аргументы, для произвольных функций -
    только первые три аргумента.
\end{frame}

\begin{frame}{Определение знаковости}
\begin{itemize}
    \item На уровне ассемблера не всегда возможно определить, является ли
        значение sink'a знаковым или беззнаковым.
    \item Для обнаружения знаковости используем следующий алгоритм:
    \begin{itemize}
        \item Идем в обратном порядке по всем условным переходам, в которых были использованы
            переменные, которые есть в sink'e
        \item Из найденных условных переходов узнаем знаковость
    \end{itemize}
    \item Также можем определить знаковость, если соответствующие входные данные
        были обработаны функцией \texttt{strto*l}.
\end{itemize}
\end{frame}

\begin{frame}{Дополнительные условия на предикат безопасности}
\begin{itemize}
    \item В аргументе \texttt{*alloc} функций хотим
        переполнить значение так, чтобы результат был меньше изначального конкретного
        значения (но не нулем)
    \item В аргументе \texttt{memcpy} и т.п. хотим
        переполнить так, чтобы результат был больше изначального конкретного
        значения
\end{itemize}
\end{frame}

\begin{frame}[fragile]{Пример Integer Overflow to Buffer Overflow}
    Входные данные: +00000000002\\
    Ответ: +01073741825 \\
    Разрядность: 32 бита
\begin{lstlisting}[language=C, basicstyle=\footnotesize, numbers=left,
                   xleftmargin=2em, numberstyle=\tiny\color{gray}]
int main() {
    int size;
    fscanf(stdin, "%d", &size);
    if (size <= 0) return 1;
    size_t i;
    int *p = malloc(size * sizeof(int));
    if (p == NULL) return 1;
    for (i = 0; i < (size_t)size; i++) {
        p[i] = 0;
    }
    printf("%d\n", p[0]);
    free(p);
}
\end{lstlisting}
\end{frame}

\begin{frame}{Juliet Dynamic}
Для проверки эффективности работы предикатов была сделана тестовая система на
основе набора тестов Juliet.
\begin{itemize}
    \item Собираем и запускаем тесты, собираем результаты TP, TN, FP, FN на
        основе вывода Sydr
    \item Проверяем на основе сгенерированного файла корректность результата с
        помощью санитайзеров
    \item Выводим отдельно результаты на основе вывода Sydr и результаты,
        верифицированные санитайзерами
\end{itemize}
\end{frame}

\begin{frame}{Результаты Juliet Dynamic}
\footnotesize
\begin{adjustwidth}{-2.5em}{-2.5em}
    \begin{tabular}{ l r | r r >{\columncolor[gray]{0.9}}r|
r r >{\columncolor[gray]{0.9}}r}\toprule
\multirow{2}{*}{\textbf{CWE}} & \multirow{2}{*}{\textbf{P=N}} &
\multicolumn{3}{c|}{\textbf{Текстовые ошибки}} &
\multicolumn{3}{c}{\textbf{Верификация}} \\
&& \textbf{TPR} & \textbf{TNR} & \textbf{ACC} & \textbf{TPR} & \textbf{TNR} & \textbf{ACC} \\
Integer Overflow & 2580 & 99.92\% & 90.89\% & 95.41\% & 98.10\% & 90.89\% & 94.50\% \\
Integer Underflow & 1922 & 99.90\% & 91\% & 95.45\% & 97.45\% & 91\% & 94.22\% \\
Int Overflow to BOF & 188 & 100\% & 100\% & 100\% & 100\% & 100\% & 100\% \\
\bottomrule
\end{tabular}
\end{adjustwidth}
\end{frame}

\begin{frame}[fragile]{FreeImage}
С помощью предикатов безопасности было найдено переполнение целого типа в
программе FreeImage при вызове функции \texttt{fseek}:


\begin{lstlisting}[language=C, basicstyle=\footnotesize,
                   xleftmargin=2em]
unsigned off_head, off_setup, off_image, i;
...
fseek(ifp, off_setup + 792, SEEK_SET);
\end{lstlisting}
\end{frame}

\begin{frame}[fragile]{xlnt}
\end{frame}

\begin{frame}[fragile]{unbound}
\end{frame}

\begin{frame}{Заключение}
\begin{itemize}
    \item Реализованы 4 предиката безопасности
    \item Получена точность 95.59\% в Juliet Dynamic на 11 CWE
    \item Найдено несколько ошибок в FreeImage
\end{itemize}
\end{frame}

\end{document}
